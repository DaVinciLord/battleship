\section{Le modèle réseau}

\subsection{Protocoles}
	\subsubsection{Protocole Réseau}
	Le protocole réseau utilisé pour que deux joueurs puissent se connecter et jouer en réseau est le protocole TCP\footnote{Transmission Control Protocol : https://tools.ietf.org/html/rfc793} qui est un protocole en mode connecté mais également fiable quant à la connexion entre deux machines et pour la réception des paquets envoyés sur le réseau. \newline
	
	Nous avons préféré le protocole TCP au protocole UDP (\footnote{User Datagram Protocol : https://tools.ietf.org/html/rfc768} car, premièrement, son implémentation est bien plus aisée. En effet, nous n'avons pas à traiter certains cas d'erreurs, notamment dans le cas d'un paquet qui soit mal formé ou imcomplet car TCP assure que celui-ci sera totalement prêt avant de l'envoyer. \newline
	Deuxièmement, malgré le fait que le protocole UDP soit plus rapide dans l'envoi des données, utiliser le protocole TCP dans notre cadre n'a pas d'impact significatif puisque, comme il sera exposé un peu plus tard, chaque joueur doit attendre la réception de données avant de continuer son traitement. Ainsi, il n'est pas, en pratique, possible de recevoir des données qui ne sont pas attendues à des moments précis.

	\subsubsection{Protocole de l'application}
		L'application doit suivre un certain protocole pour que les deux joueurs puissent communiquer correctement et sans problèmes majeurs. Il a donc été décidé de suivre un type de protocole classique qui est celui du question/réponse. Néanmoins, nous n'avons pas choisi d'implémenter de serveur d'hébergement centralisant toutes les parties et dirigeant toutes les actions des clients, nous avons opté pour un échange de type pair à pair, c'est à dire que les deux joueurs doivent jouer les deux rôles: le processus serveur et le processus client, chacun à des temps déterminés et non mutables. \newline
		Nous expliquerons dans cette partie la connexion entre les deux joueurs ainsi que comment se passe l'échange de données.
		\subsubsubsection{Connexion des joueurs}
			Comme dit plus haut, chaque joueur doit jouer deux rôles bien distincts, celui du processus serveur comme celui du processus client. Dans ce cas-ci, il est donc bon de noter l'ordre des différentes opérations effectuées entre un joueur qui va accueillir la partie et celui qui va en être l'invité, nous allons donc expliciter quelles actions font parties du processus client et lesquelles font parties du processus serveur. \newline
			
			Les deux joueurs vont, en premier lieu, créer et lancer leur serveur respectif, en spécifiant l'adresse de la machine en format IPv4, un port d'écoute (nous vérifions qu'il soit compris entre 1024 et 65535), ainsi qu'un nombre de connexions autorisées qu'on nommera "backlog". \newline
			
			Ensuite, le joueur hôte met son serveur en attente d'une demande de connexion via sa socket d'écoute (il est donc considéré en tant que processus serveur) et, de son côté, le joueur invité (considéré comme le processus client à ce moment) va se connecter à l'adresse du joueur hôte. Cela aura pour effet, côté processus serveur, de créer une socket de service par laquelle le joueur passera pour communiquer avec le joueur invité. \newline
			
			À cette étape, il reste à faire la connexion inverse, c'est à dire que le joueur hôte (qui est maintenant le processus client) doit faire une demande de connexion à l'adresse du serveur du joueur invité (devenu le processus serveur) et, une fois la connexion acceptée par la socket d'écoute du processus serveur, une nouvelle socket de service est créée pour que le joueur invité puisse communiqué avec le joueur hôte.\newline
			
			Suite à ces différentes étapes, la connexion entre le joueur hôte et le joueur invité est établie, les échanges de données peuvent donc être effectuée.

		\subsubsubsection{Échanges des données}

\subsection{Gestion des erreurs}
	La gestion des erreurs est faites sur plusieurs niveaux d'abstraction dans notre projet. En effet, le premier niveau d'abstraction est le fait d'utiliser le protocole TCP/IP, car celui-ci, en opposition au protocole UDP, en plus de recevoir le paquet fait une vérification afin de valider le paquet avant de le transmettre. De ce fait, une partie des vérifications est déjà effectué par ce biais.\newline
	
	Le second niveau d'abstraction est lié au langage que nous avons choisi. En effet, Java implémente les protocoles TCP et UDP. Hors, Java effectue aussi des vérifications quand à la validité des données et permet facilement de vérifier si le paquet reçu est valide ou non, par le biais des Exceptions présente dans l'API de base.\newline
	
	Au final, nous avons implanté notre propre protocole de validité des données, afin d'être sur de ce que nous recevons. Comme vous avez pu le lire précédemment, chacune de nos données envoyées possède un format propre a notre protocole et si, lors d'un envoie de donnée, nous n'avons pas le bon type de données, alors nous gérons le problème en indiquant qu'un problème est survenue lors de l'envoie des données, ce qui n'est pas normal et qui indique un problème au niveau du programme. De ce fait, nous fermons donc proprement l'application afin d'éviter tout problèmes lors d'une exécution prolongée du logiciel.
