\section{Modèle de jeu}

\subsection{Le plateau de jeu}

	Dans le cadre de ce projet, nous avons choisi de donner aux joueurs la possibilité de jouer différemment à la bataille navale, notamment de jouer en trois dimensions.\newline
	
	Pour cela, nous avons réfléchi puis développé un modèle capable de s'adapter aux besoins des joueurs. En effet, celui-ci peut décider de toutes les tailles, longueur, largeur ou encore profondeur, ainsi qu'un nombre de dimensions quelconques. \newline
	
	Le plateau de jeu permet donc de jouer en une dimension (soit sur la longueur ou soit sur la largeur), en deux dimensions (longueur et largeur), en trois dimensions nous ajoutons la profondeur pour enfin pouvoir jouer en plus de trois dimensions.\newline
	

\subsection{Les intelligences artificielles}
	Il nous à été demandé de pouvoir jouer à la bataille navale en local. 
De ce fait, nous avons due développer une Intelligence Artificiel (IA), car jouer à deux sur un même ordinateur présente de nombreux risques de triches.\newline
	Notre IA se compose de 3 niveaux de difficultés différents, allant de facile à difficile.\newline
	Nous allons donc vous expliquez le fonctionnement de chacune de ces 3 IA.\newline

\subsubsection{IA Facile}
    L'IA facile fut la première intelligence artificielle programmée. Son cheminement est simple, elle consiste simplement à choisir aléatoirement une case, en vérifiant bien entendu que celle-ci n'est pas déjà était tirée au préalable, puis de tirer dessus, qu'importe l'avancement de la partie.\newline
    Le choix de l'aléatoire est fait de la façon suivante :\newline
    
    \begin{verbatim}
        Fonction EasyAdvisor :
        Entrée : Board un tableau de state contenant les resultats de tous les tirs.
                 Dim le tableau des tailles des dimension de Board.
        Sortie : Coord une Coordonnée d’une case non touchée.
        Début
            Faire {
                Entier distance = Alea(0, taille(Board)) ;
                Entier longueur = taille(Board) ;
                Pour (n allant de 0 à taille(Dim)) {
                    longueur = longueur / Dim[taille(dim) - 1 – n) ;
                    Ajout(Coord, taille(dim) - 1 – n, distance / longueur) ;
                    Distant = distance % longueur
                }
            } tant que (Board[Coord] != NON_CIBLÉ) ;
            Retourner Coord ;
        Fin
    \end{verbatim}
        
\subsubsection{IA Moyen}
	L'IA moyen, deuxième IA développée, fonctionne de manière similaire à l'IA facile, à ceci près qu'une fois un bateau touché l'IA passe en mode "chasse".\newline
	Le mode chasse indique à l'IA de tirer sur les quatres directions(nord, sud, est et ouest) entourant la case touchée et, si une nouvelle case est touchée, repart de cette dernière pour ainsi continuer jusqu'à avoir coulé le bateau.\newline
	Elle repart alors dans sa routine de ciblage.\newline
	L'algorithme est présenté ci-dessous:
    
    \begin{verbatim}
        Procedure MediumAdvisor :
            Entrée : Board un tableau de state contenant les résultats
            		 de tous les tirs.
                     Dim le tableau des tailles des dimension de Board.
                     Coe une liste de case.
            Sortie : Coord une Coordonnée d’une case non touchée.
            
            Début
                Si estVide(coe) {
                    Coodinate c = EasyAdvisor(Board, Dim) ;
                    SI (Tir(enemy, c) != MANQUÉ) {
                        AjouterCasesAdjacentes(coe, coord) ;
                    }
                    Retourner Coord ;
                }
                Sinon {
                    Si (Board[Coe[0]] == NON_CIBLÉ) {
                        Coord = coe[0] ;
                        Retirer(coe, 0) ;
                        Retourner Coord ;
                    }
                    MediumAdvisor(Board, Dim, Coe);
                }
            Fin
    \end{verbatim}

\subsubsection{IA Difficile}
    L'IA difficile, troisième IA développée, fonctionne de manière similaire à l'IA normale, c'est à dire qu'elle effectue sa routine de "ciblage" jusqu'à toucher un bateau, puis passe en mode "chasse" une fois un bateau touché.
    La différence avec l'IA normale est que cette IA ne tire pas de case adjacente l'une à l'autre. En effet, suivant la taille minimal du bateau ennemi existant au début de partie, afin qu'elle ne tire que toute les tailles du bateau minimal.
    Par exemple, dans le cas ou le plus petit bateau est un batea de 2, l'IA va donc écarter ses tirs de 2 cases d'écart à chaque fois. 
    \begin{verbatim}
            Procedure MediumAdvisor :
            Entrée : Board un tableau de state contenant les resultats de tous les tirs.
                     Dim le tableau des tailles des dimension de Board.
                     Coe une liste de case.
                     Minsize la taille du plus petit bateau present sur la grille
            Sortie : Coord une Coordonnée d’une case non touchée.

            Si estVide(coe) {
                Faire {
                    Entier distance = Alea(0, taille(Board)) ;
                    Entier longueur = taille(Board) ;
                    Entier mod = 0 ;
                    Pour (n allant de 0 à taille(Dim)) {
                        Mod = 0 ;
                        longueur = longueur / Dim[taille(dim) - 1 – n) ;
                        Ajout(Coord, taille(dim) - 1 – n, distance / longueur) ;
                        Distant = distance % longueur
                        Mod = (mod + distance /longueur) % minsize;
                    }
                } tant que (mod != 0 || Board[Coord] != NOTAIMED) ;

                SI (Tir(enemy, coord) != MISSED) {
                    AjouterCasesAdjacentes(coe, coord) ;
                }   
                Retourner Coord ;
            }
            Sinon {
                Si (Board[Coe[0]] == NOTAIMED {
                    Coord = coe[0] ;
                    Retirer(coe, 0) ;Retourner Coord ;
                }
            }
            MediumAdvisor(Board, Dim, Coe) ;
        Fin Procedure
    \end{verbatim}