\section{Modèle de jeu}

\subsection{Le plateau de jeu}

	Dans le cadre de ce projet, nous avons choisi de donner aux joueurs la possibilité de jouer différemment à la bataille navale, notamment de jouer en trois dimensions.\newline
	
	Pour cela, nous avons réfléchi puis développé un modèle capable de s'adapter aux besoins des joueurs. En effet, celui-ci peut décider de toutes les tailles, longueur, largeur ou encore profondeur, ainsi qu'un nombre de dimensions quelconques. \newline
	
	Le plateau de jeu permet donc de jouer en une dimension (soit sur la longueur ou soit sur la largeur), en deux dimensions (longueur et largeur), en trois dimensions nous ajoutons la profondeur pour enfin pouvoir jouer en plus de trois dimensions.\newline
	
	\subsubsection{Définition récursive des Boards}
		Nous avons uns structure donnée qui se définit récursivement de la manière suivante : \newline
		\begin{itemize}
		\item Une case d'un Board est un Board à zéro dimensions.\newline
		\item Un Board à N dimensions, dont les tailles dans chaque dimension sont (TN, TN-1, … , T1) où les Ti sont les tailles, est composé de TN Boards à N - 1 dimensions, dont les tailles dans chaque dimension sont (TN-1, … , T1). 
		\end{itemize}
			
	Notez bien que les Boards à N - 1 dimensions ont exactement les même tailles pour chaque dimension. Ainsi, un Board à deux dimensions sera toujours représentable comme un rectangle, un Board à trois dimensions peut-être visualisé comme un pavé droit et ceux des dimensions supérieures suivent le même principe, bien qu'ils soient plus difficile à représenter.
\newline

	On peut aussi voir le Board comme un arbre particulier avec les propriétés suivantes : \newline
		\begin{itemize}
			\item La distance (plus court chemin de la racine à une feuille) et la hauteur (plus long chemin de la racine à une feuille) de l'arbre sont égales.\newline
			\item Tous les nœuds d'un même niveau ont le même nombre de descendants.
		\end{itemize}

	Cette structure récursive nous permet d'instancier un Board de n'importe quel nombre de dimensions en lui passant tout simplement un tableau d'entier classique contenant les tailles dans chaque dimension.

	\subsubsection{Remplissage des Boards}

	Pour remplir un tableau classique en programmation, on utilise généralement autant de boucles imbriquées qu'il y a de dimensions au tableau. Dans le cas du Board, ce n'est pas possible à notre connaissance, puisque le nombre de dimensions est inconnu à la compilation et qu'il faudrait un nombre variable de boucles imbriquées dans le code.\newline
	
	Nous avons donc créé un itérateur de Boards à nombre de dimensions zéro, implémenté dans la classe interne de Board, DimZeroIterator qui donne un par un les Boards à zéro dimensions qui sont les cases du Board. \newline

	Cet itérateur est également construit de manière récursive. Il contient les DimZeroIterator correspondants aux Boards fils du Board qu'il itère et les méthodes next() et hasNext() appelent leurs homologues sur l'itérateur fils en cours de parcours.

	Si on voit le Board comme un arbre, cet itérateur donne les cases dans l'ordre où un parcours en profondeur donnerait les feuilles.

	\subsubsection{Accès aux cases}

	L'accès en lecture et en écriture d'une case en particulier se fait grâce aux objets Coordinates.
Ces objets sont techniquement des tableaux d'entiers non mutables, dont on peut tester l'égalité avec la méthode equals() redéfinie. L'accès à une case de ce tableau se fait avec la méthode get(int index), à la façon des listes, mais il est itérable comme un tableau classique.\newline

	Lors d'un appel à getItem() ou setItem() sur un Board, chaque composante de la Coordinates fournie en paramètre sert à déterminer sur quel Board de dimension inférieure il faut effectuer l'appel récursif. \newline

	Les objets Coordinates sont également utilisés pour instancier les Boards en fournissant le nombre de dimensions et la taille pour chaque dimension.

	\subsubsection{Détails de l'implantation}

	Les algorithmes d'accès aux cases de Board sont récursifs terminaux; c'est à dire que lors de la compilation, le compilateur (si il y a compilation, ce qui est notre cas ici) peut remplacer ces appels récursifs par une simple boucle itérative. Cela s'explique par le fait que le rappelle a soi-même soit la dernière instruction a être évaluée. De ce fait, cela permet d'économiser de la mémoire lors des appels a ces fonctions, car seul l'adresse de l'appel de la fonction doit être sauvegarder sur la pile d'exécution. \newline
	Structures des objets :

	\begin{alltt}
	{\bf Structure} Board :
    	{\bf Champ} Object : item;
    	{\bf Champ} Liste de Board : boards;
	{\bf fin Structure}
		
	{\bf Structure} Coordinates :
    	{\bf Champ} Tableau d'entiers : coord;
	    {\bf Champ} Entier : length;
	{\bf fin Structure}
	\end{alltt}


	\subsubsubsection{Algorithmes de Coordinates}
	
	\begin{alltt}
{\bf Algo} Coordinates creerCoordinates(Tableau d'entier t) :
    coord <- copie de t;
	length <- longueur de t;
{\bf finAlgo}
			
{\bf Algo} Entier get(Coordinates c, Entier index) :
    renvoyer c.coord[index];
{\bf finAlgo}
	\end{alltt}

		
	\newpage
	\subsubsubsection{Algorithmes de Board}

	\begin{alltt}
{\bf Algo} Board creerBoard(Coordinates c) :
    renvoyer creerBoard(c, 0);
{\bf finAlgo}
		
{\bf Algo} Board creerBoard(Coordinates c, entier index) :
    Board b;
    b.item <- null;
    {\bf si} index = c.length {\bf alors}
        renvoyer b;
    {\bf finSi}
    b.boards <- nouvelle Liste de Board;
    {\bf pour} entier k de 0 à c.get(index) {\bf faire}
        b.boards[k] <- creerBoard(c, index + 1);
    {\bf finPour}
    renvoyer b;
{\bf finAlgo}
		
{\bf Algo} Object getItem(Board b, Coordinates c) :
    renvoyer getItem(b, c, 0);
{\bf finAlgo}
		
		
{\bf Algo} Object getItem(Board b, Coordinates c, Entier index) :
    {\bf si} index = c.length {\bf alors}
        renvoyer b.item;
    {\bf finSi}
    renvoyer getItem(b.boards.get(c.get(index)), c, index + 1);
{\bf finAlgo}
		
{\bf Algo} voidsetItem(Board b, Coordinates c, Object item) :
    setItem(b, c, 0, item);
{\bf finAlgo}
		
{\bf Algo} void setItem(Board b, Coordinates c, Entier index, Object item) :
    {\bf si} index = c.length {\bf alors}
        b.item <- item;
    {\bf finSi}
    setItem(b.boards.get(c.get(index)), c, index + 1, item);
{\bf finAlgo}
	\end{alltt}

\subsection{Les bateaux en dimensions variables}
	Les navires sont représentés par la classe Ship. Ils ont une longueur, un nom et contiennent les coordonnées des cases du Board qu'ils occupent lorsqu'ils sont placés. \newline

	Dans une bataille navale classique, chaque navire est disposé verticalement ou horizontalement. Dans notre modèle à nombre variable de dimensions, il faut généraliser cette règle pour qu'elle fonctionne dans tous les cas.

\subsubsection{Définition d'un bon placement}

	Pour être bien placés, les Ship doivent respecter les conditions suivantes :
	\begin{itemize}
		\item Bon alignement : les cases occupées par un Ship doivent avoir des coordonnées dont toutes les composantes sauf une sont égales, et les composantes qui ne le sont pas doivent se suivre. \newline
		Exemple : Dans un Board en quatre dimensions, un porte-avion (de longueur 5) occupe les cases (0,2,1,3), (0,2,2,3), (0,2,3,3), (0,2,4,3), (0,2,5,3). Ce placement respecte la condition de bon alignement car les premières, deuxièmes et quatrièmes composantes des coordonnés sont respectivement égales entre-elles, et que les troisièmes composantes se suivent. \newline
		
		\item Non dépassement : les cases occupées par un Ship sont celles contenues dans un Board. \newline
		Exemple : Dans un Board de dimensions (5,5,5,5), le porte-avion placé précédemment dépasse par sa dernière coordonnée, dont la troisième composante n'est pas comprise entre 0 et 4. \newline
		
		\item Bonne longueur : un Ship doit occuper exactement le même nombre de cases que sa longueur. \newline
		
		\item Non-Chevauchement : une case ne peut être occupée que par un seul Ship.
	\end{itemize}

\subsubsection{Tir sur les Ships}

	Les tirs sur les Ships se font via le Board sur lequel ils sont disposés. Pour cela, le type générique Board est paramétré soit par le type Case, soit par le type State. \newline
	
	Les Board<Case> de chaque joueur sont ceux sur lesquels ils placent leurs navires, et les Board<State> représentent la grille de tir des joueurs, là où ils peuvent voir où ils ont tirés et pour quel résultat. \newline
	
	State est une énumération représentant le statut de tir de chaque case d'un Board<State>. Il existe quatre états possibles :
	\begin{itemize}
		\item NOT$\_$AIMED : cette case n'a pas été visée par un tir.
		\item MISSED : il y a eu un tir sur cette case, mais il n'y avait pas de Ship adverse sur la case correspondante du Board<Case> du joueur adverse.
		\item HIT : Il y a eu un tir sur cette case, et un Ship adverse a été touché.
		\item SUNK : Même chose que HIT, avec en plus l'information que le Ship touché a été coulé par ce tir.
	\end{itemize}

	Les objets Case ont pour rôle de stocker deux informations : la présence d'un Ship et si la case a déjà reçu un tir. Ces deux informations suffisent à faire la distinction entre les états NOT$\_$AIMED, MISSED et HIT/SUNK. \newline
	
	Un objet Case sert également à relayer un tir jusqu'au Ship qui l'occupe, s'il y en a un, et à s'assurer qu'on ne tire pas deux fois au même endroit. Cette condition rend non-nécessaire le stockage dans les Ship des endroits précis où ils ont été touchés : un Ship qui a été touché un nombre de fois égal à sa longueur a forcément été touché sur toutes les cases qu'il occupe, et on sait ainsi qu'il a été coulé.\newline	
	
	Voici ce qui conclut la partie sur le modèles en tant que tels, vous avez vu comment est généré notre plateau de jeu ainsi, la façon dont on place nos bateaux sur cette grille de jeu ainsi que la façon dont se passe les tirs. Nous allons donc maintenant voir les intelligences artificielles développer dans le cadre du projet.
	
	\newpage

\subsection{Les intelligences artificielles}
	Il nous à été demandé de pouvoir jouer à la bataille navale en local. 
De ce fait, nous avons due développer une Intelligence Artificiel (IA), car jouer à deux sur un même ordinateur présente de nombreux risques de triches.\newline
	Notre IA se compose de 3 niveaux de difficultés différents, allant de facile à difficile.\newline
	Nous allons donc vous expliquez le fonctionnement de chacune de ces 3 IA.\newline

\subsubsection{IA Facile}
    L'IA facile fut la première intelligence artificielle programmée. Son cheminement est simple, elle consiste simplement à choisir aléatoirement une case, en vérifiant bien entendu que celle-ci n'est pas déjà était tirée au préalable, puis de tirer dessus, qu'importe l'avancement de la partie.\newline
    Le choix de l'aléatoire est fait de la façon suivante :\newline
    
\begin{alltt}
{\bf Fonction} EasyAdvisor :
	{\bf Entrée} : Board un tableau de state contenant les resultats de tous les tirs.
          Dim le tableau des tailles des dimension de Board.
    {\bf Sortie} : Coord une Coordonnée d’une case non touchée.
    {\bf Début}
        {\bf Faire}
            Entier distance = Alea(0, taille(Board)) ;
            Entier longueur = taille(Board) ;
            {\bf Pour} (n allant de 0 à taille(Dim))
                longueur = longueur / Dim[taille(dim) - 1 – n) ;
                Ajout(Coord, taille(dim) - 1 – n, distance / longueur) ;
                Distant = distance % longueur
            {\bf finPour}
        {\bf Tant que} (Board[Coord] != NON_CIBLÉ) ;
        Retourner Coord ;
    {\bf Fin}
{\bf Fin Procedure} 
\end{alltt}
        
\newpage        
        
\subsubsection{IA Moyen}
	L'IA moyen, deuxième IA développée, fonctionne de manière similaire à l'IA facile, à ceci près qu'une fois un bateau touché l'IA passe en mode "chasse".\newline
	Le mode chasse indique à l'IA de tirer sur les quatres directions(nord, sud, est et ouest) entourant la case touchée et, si une nouvelle case est touchée, celle-ci repart de cette dernière pour ainsi continuer jusqu'à avoir coulé le bateau.\newline
	Elle repart alors dans sa routine de ciblage.\newline
	L'algorithme est présenté ci-dessous:
    
\begin{alltt}
{\bf Procedure} MediumAdvisor :
    {\bf Entrée} : Board un tableau de state contenant les résultats de tous les tirs.
              Dim le tableau des tailles des dimension de Board.
              Coe une liste de case.
    {\bf Sortie} : Coord une Coordonnée d’une case non touchée.

    {\bf Début}
        {\bf Si} estVide(coe) 
            Coodinate c = EasyAdvisor(Board, Dim) ;
            {\bf SI} (Tir(enemy, c) != MANQUÉ)
                AjouterCasesAdjacentes(coe, coord) ;
            {\bf FinSi}
            Retourner Coord ;
        {\bf Sinon} 
            {\bf Si} (Board[Coe[0]] == NON_CIBLÉ)
                Coord = coe[0] ;
                Retirer(coe, 0) ;
                Retourner Coord ;
            {\bf FinSi}
            MediumAdvisor(Board, Dim, Coe);
        {\bf FinSi}
    {\bf Fin}
{\bf Fin Procedure}            
\end{alltt}

\newpage

\subsubsection{IA Difficile}
    L'IA difficile, troisième IA développée, fonctionne de manière similaire à l'IA normale, c'est à dire qu'elle effectue sa routine de "ciblage" jusqu'à toucher un bateau, puis passe en mode "chasse" une fois un bateau touché.
    La différence avec l'IA normale est que cette IA ne tire pas de case adjacente l'une à l'autre. En effet, suivant la taille minimal du bateau ennemi existant au début de partie, afin qu'elle ne tire que toute les tailles du bateau minimal.
    Par exemple, dans le cas ou le plus petit bateau est un bateau de 2, l'IA va donc écarter ses tirs de 2 cases d'écart à chaque fois.
    
\begin{alltt}
{\bf Procedure} HardAdvisor :
    {\bf Entrée} : Board un tableau de state contenant les résultats de tous les tirs.
              Dim le tableau des tailles des dimension de Board.
              Coe une liste de case.
              Minsize la taille du plus petit bateau présent sur la grille
    {\bf Sortie} : Coord une Coordonnée d’une case non touchée.

    {\bf Début}
        {\bf Si} estVide(coe)
            {\bf Faire}
                Entier distance = Alea(0, taille(Board)) ;
                Entier longueur = taille(Board) ;
                Entier mod = 0 ;
                {\bf Pour} (n allant de 0 à taille(Dim))
                    Mod = 0 ;	
                    longueur = longueur / Dim[taille(dim) - 1 – n) ;
                    Ajout(Coord, taille(dim) - 1 – n, distance / longueur) ;
                    Distant = distance % longueur
                    Mod = (mod + distance /longueur) % minsize;
                {\bf FinPour}
            {\bf Tant Que}(mod != 0 || Board[Coord] != NOTAIMED) ;

            {\bf SI} (Tir(enemy, coord) != MISSED)
                AjouterCasesAdjacentes(coe, coord) ;
            {\bf FinSi}   
            Retourner Coord ;
        {\bf Sinon}
            {\bf Si} (Board[Coe[0]] == NOTAIMED)
                Coord = coe[0] ;
                Retirer(coe, 0) ;
                Retourner Coord ;
            {\bf FinSi}
        {\bf FinSi}
        MediumAdvisor(Board, Dim, Coe) ;
    {\bf Fin}
{\bf Fin Procedure}
\end{alltt}
    
    Voici ce qui conclut notre première partie présentant le modèle de notre application, nous avons donc vu dans un premier temps le fonctionnement de notre grille de jeu, qui est donc une grille modulable à un nombre de dimension infini. Nous avons ensuite expliqué comment fonctionne nos bateaux, et la façon dont on tire dessus. Puis pour finir, nous avons montré le fonctionnement de nos IA ainsi que les différences entres les 3 niveaux de difficultés.
    Nous allons donc maintenant passer sur la seconde partie du projet qui est celle du réseau.