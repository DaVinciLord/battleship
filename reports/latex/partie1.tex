\section{Modèle de jeu}


\subsection{Le plateau de jeu}

	Dans le cadre de ce projet, nous avons choisi de donner aux joueurs la possibilité de jouer différemment à la bataille navale, notamment de jouer en trois dimensions.\newline
	
	Pour cela, nous avons réfléchi puis développé un modèle capable de s'adapter aux besoins des joueurs. En effet, celui-ci peut décider de toutes les tailles, longueur, largeur ou encore profondeur, ainsi qu'un nombre de dimensions quelconques. \newline
	
	Le plateau de jeu permet donc de jouer en une dimension(soit sur la longueur ou soit sur la largeur), en deux dimensions (longueur et largeur), en trois dimensions nous ajoutons la profondeur pour enfin pouvoir jouer en plus de trois dimensions.


\subsection{Les intelligences artificielles}
%
%\foreignlanguage{english}{Do you speak French? Does anybody here speak french?}
%
%
%\subsection{Lites}
%
%\begin{itemize}
%\item Liste classique ;
%\item un élément ;
%\item et un autre élément.
%\end{itemize}
%\vspace{\parskip} % espace entre paragraphes
%
%\begin{enumerate}
%\item Une liste numéroté
%\item deux
%\item trois
%\end{enumerate}
%\vspace{\parskip}
%
%\begin{description}
%\item[Description] C'est bien pour des définitions.
%\item[Deux] Ou pour faire un liste spéciale.
%\end{description}
%\vspace{\parskip}
%
%
%\subsection{Références}
%
%Voici une référence à l'image de la figure \ref{bloghiko} page \pageref{bloghiko} et une autre vers la partie \ref{p2} page \pageref{p2}.
%
%On peut citer un livre\,\up{\cite{lpp}} et on précise les détails à la fin du rapport dans la partie références.
%
%
%\subsection{Note de bas de page}
%
%Voici une note\,\footnote{Texte de bas de page} de bas de page.
%Une deuxième\,\footnotemark{} déclarée différemment.
%La même note\,\footnotemark[\value{footnote}].
%
%\footnotetext{Il a deux références vers cette note}
%
%
%\subsection{Figure}
%
%\begin{figure}[!ht]
%    \center
%    \caption{BlogHiko | taille original}
%    \label{bloghiko}
%\end{figure}
%
%
%
