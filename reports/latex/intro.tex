\section*{Introduction} % Pas de numérotation
\addcontentsline{toc}{section}{Introduction} % Ajout dans la table des matières
	De nos jours, il est possible de jouer à des jeux de société même avec des personnes qui ne sont pas dans la même pièce grâce aux différents protocoles réseaux. De plus en plus de jeux sont donc implémentés et utilisent ces technologies pour permettre de jouer en réseau, c'est à dire, de pouvoir jouer entre deux machines distantes. \newline

	De cette constatation, le besoin est venu d'implémenter un classique des jeux de société : la bataille navale (ou "touché coulé") qui est un jeu se jouant à deux joueurs dont le but est de couler tous les bateaux de l'adversaire, disposés sur un plateau de jeu. La bataille navale pouvant se jouer avec plusieurs règles, nous avons opté pour "le tour par tour" : chaque joueur joue une et une seule fois puis donne la main à son adversaire. \newline
	
	Ce projet a pour objectif de disposer de deux modes de jeu: un en local permettant de jouer contre une intelligence artificielle et l'autre permettant de jouer contre un adversaire humain en réseau.\newline
	
	Dans ce rapport, nous allons présenter les différents points de l'application, notamment en décrivant le modèle du jeu, ensuite la partie réseau, puis l'interface homme machine permettant à l'utilisateur de jouer dans des conditions correctes, enfin nous exposerons dans une dernière partie les différentes connaissances que ce projet a apporté, les difficultés rencontrées ainsi que l'évolution possible de ce projet.
